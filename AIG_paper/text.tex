\section{Non-Monotonic Logics}
The basis was laid out by \cite{mcdermott_non-monotonic_1980}.

\subsection{Introduction to Non-Monotonic Logics}
text\\
"The purpose of non-monotonic inference rules is not to add certain knowledge where there is none, but rather to guide the selection of tentatively held beliefs in the hope that fruitful investigations and good guesses will result. This means that one should not \textit{a priori} expect non-monotonic rules to derive valid conclusions independent of the monotonic rules. Rather one should expect to be led to a set of beliefs which while perhaps eventually shown incorrect will meanwhile coherently guide investigations."\\
\cite[p. 46]{mcdermott_non-monotonic_1980}

\subsection{Understanding Inconsistencies in Logic Systems}
text

\subsection{Challenges Specific to Non-Monotonic Logics}
text

\section{Measuring Inconsistency}
Analyze the distribution of inconsistency within a knowledge base using the Shapely measure \cite{hunter_measure_2010} in the classical case. Measures based on minimal inconsistent sets of a knowledge base \(\mathcal{K}\) \cite{jabbour_mis_2016} were proposed by \cite{ulbricht_measuring_2018}.

\subsection{Measures of Inconsistency in Non-Monotonic logics}
text

\subsection{Case Studies and Applications}
text

\subsection{Future Directions}
text

\iffalse
    KE3 - MWV
    Manche Dinge wissen wir nicht, weil wir sie nie gelernt oder schon wieder vergessen haben; einiges ist generell unbekannt, zu anderen Informationen haben wir vielleicht keinen Zugang. Oft fehlt uns auch die Zeit, um genügend Information zu beschaffen, und wir müssen auf der Basis des Wenigen, das wir wissen, unverzüglich Entscheidungen treffen.
    • Es ist schlichtweg unmöglich, eine reale Situation im logischen Sinne vollständig zu beschreiben. Ebensowenig lassen sich alle nur erdenklichen Ausnahmen zu einer Regel im allgemeinen Sinne anführen. Und selbst wenn dies (in einem begrenzten Rahmen) realisierbar wäre, so wurde doch die Uberprüfung aller dieser Details viel zu lange dauern.
    • In vielen Dingen abstrahieren wir daher von unwichtig erscheinenden Aspekten und konzentrieren uns auf das Wesentliche. Unvollständige Information kann also auch sinnvoll und beabsichtigt sein.
    • Wir können Situationsmerkmale übersehen oder falsch wahrnehmen.
    • Über Dinge, die in der Zukunft liegen, können wir nur spekulieren.
    • Natürliche Sprache ist oft kontextabhängig und selten ganz eindeutig. Hier sind wir auf unsere Fähigkeit angewiesen, Dinge im Zusammenhang zu interpretieren. Missverständnisse können dabei allerdings nie ganz ausgeschlossen werden.

    G. Antoniou beschreibt in seinem Buch “Nonmonotonic reasoning” [Ant97] dieses Dilemma sehr treffend:
    Try to explain what a default rule is and everybody will understand because everybody has come across them; try to explain that an extension is a solution of the [fixpoint] equation ΛT (E) = E, and most people will flee in panic!
\fi
