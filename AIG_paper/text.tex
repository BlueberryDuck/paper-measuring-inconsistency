\section{Introduction}
text

\section{Inconsistency Measurements}
Not all inconsistent knowledge bases contain no information at all or the same amount of, just because they are inconsistent as a whole. This can be deduced as two inconsistent knowledge bases can provide different conclusions, implying they contain different information \cite{bertossi_approaches_2005}.

A binary measure, commonly used for example in classical inference, only provides information on whether a knowledge base is consistent or inconsistent. Other logics that can work with inconsistent knowledge bases would profit from more fine-grained measures.
One example of such logic is the quasi-classical logic \cite{grant_measuring_2006} which extends the first-order logic to enable the quantification of equality and more importantly inconsistency in a knowledge base by determining "degrees of inconsistency".

The fundamental idea of an inconsistency measure is to provide a function \(\mathcal{I}\) that when applied to a knowledge base \(\mathcal{K}\) is growing in number the more inconsistencies there are in \(\mathcal{K}\). One of the most popular approaches for such a measure is using the number of minimal inconsistent subsets \cite{hunter_measuring_2008}. This approach "counts" the minimal number of formulas needed to depict the inconsistency of the entire set, while also taking into account the proportion of the language that is touched by the inconsistency.\\
So when using a set of all minimal inconsistent subsets of a knowledge base \(I_{\min}(\mathcal{K})\) one can define an inconsistency measure as \(\mathcal{I}_{\text{MI}}(\mathcal{K}) = \left| I_{\min}(\mathcal{K}) \right|\). This simple approach comes with the advantage of already meeting a number of the basics that an inconsistency measure needs to fulfill, most notably \(\mathcal{I}_{\text{MI}}(\mathcal{K}) = 0\) if \(\mathcal{K}\) is consistent.

Many other inconsistency measures expand on that idea. \cite{jabbour_mis_2016} uses minimal inconsistent subset partitions to be able to differentiate within the subsets how strong conflicts between them are. This means that not all subsets count equally towards the inconsistency score, which provides an even finer measure.

\section{Definition Answer Set Programming}
Possibly defined here for use as an example onwards.
\\
"Some examples of such formalisms are, e.g. [...] answer set programming \cite{gelfond_logic_2002} [...]"

\section{Strong Inconsistency}
Let \(\mathcal{K}\) be the knowledge base of an implicit logic. \(I(\mathcal{K})\) denotes the set of all inconsistent subsets, \(I_{\min}(\mathcal{K})\) the set of all minimal inconsistent subsets, \(C(\mathcal{K})\) the set of all consistent, and \(C_{\max}(\mathcal{K})\) the set of all maximal \(\mathcal{K}\)-consistent subsets of \(\mathcal{K}\). In weakly monotonic logics if a knowledge base \(\mathcal{K} \subseteq \mathcal{K}'\) is inconsistent then so is \(\mathcal{K}'\). Additionally, a specific duality between minimal inconsistent and maximal consistent sets holds true.

Let \(\mathcal{M}\) be a set of sets. We call \(\mathcal{H}\) a hitting set of \(\mathcal{M}\) if \(\mathcal{H}\) and all subsets \(M\) have a common intersection (\(\mathcal{H} \cap M \neq \emptyset | M \in \mathcal{M}\)). A minimal hitting set \(\mathcal{H}\) of \(\mathcal{M}\) further requires that no elements in \(\mathcal{H}\) can be dropped without losing its hitting set characteristic (\(\mathcal{H}' \subsetneq \mathcal{H}\) implies \(\mathcal{H}'\) is not a hitting set of \(\mathcal{M}\)). The MinHS duality \cite{reiter_theory_1987} implies \(\mathcal{H}\) is a minimal hitting set of \(I_{\min}(\mathcal{K})\) only if the knowledge base without the elements of \(\mathcal{H}\) forms the maximal consistent subset (\(\mathcal{K} \backslash \mathcal{H} \in C_{\max}(\mathcal{K})\)).

In non-monotonic logics neither the statement that \(\mathcal{K}'\) must be inconsistent, nor the hitting set duality has to be true because additional information can resolve inconsistency in a knowledge base, which opens up the possibility of consistent knowledge bases to contain inconsistent subsets. For this reason, a refined notion of inconsistent subsets of a knowledge base was defined by \cite{brewka_strong_2017}.

Following this definition, a subset of a knowledge base \(\mathcal{K}\) is strongly \(\mathcal{K}\)-inconsistent if all its supersets within the knowledge base are inconsistent as well. This introduces the denotations \(SI(K)\) as the set of all strongly \(\mathcal{K}\)-inconsistent subsets of \(\mathcal{K}\) and \(SI_{\min}(\mathcal{K})\) as the set of all minimal strongly \(\mathcal{K}\)-inconsistent subsets of \(\mathcal{K}\).

This opens up the possibility of exploring the measurement of inconsistency in non-monotonic logics, a subject that has only been examined comparatively recently \cite{ulbricht_measuring_2018} \cite{brewka_strong_2019} \cite{ulbricht_handling_2020}.

\section{Three Measures based on Minimal Inconsistent Sets}
"We consider generalized versions of three measures based on minimal inconsistent sets" in Section 2 of \cite{ulbricht_measuring_2018} and Section 3 of \cite{ulbricht_handling_2020}
\\
"a subset of a knowledge base \(\mathcal{K}\) is strongly inconsistent if all its supersets within \(\mathcal{K}\) are inconsistent as well. Intuitively, one can think of a conflict that cannot be resolved by formulas in \(\mathcal{K}\) itself"
\\
"classical and strong inconsistency coincide whenever our logic is monotonic \cite{brewka_strong_2019}"
\\
"In particular, the notions coincide for monotonic logics, and the existence of a strongly inconsistent subset of \(\mathcal{K}\) is a necessary and sufficient condition for the inconsistency of \(\mathcal{K}\) itself. Moreover, removing from \(\mathcal{K}\) any minimal hitting set of \(SI_{\min}(\mathcal{K})\) yields a maximal consistent subset of \(\mathcal{K}\), which is also known as the hitting set duality in classical logics \cite{reiter_theory_1987}." See \cite{brewka_strong_2017}
\\
"Assume an arbitrary but fixed logic \(L\). In classical inconsistency measurement, minimal inconsistent subsets of a knowledge base play an important role since they can be seen as the 'atomic conflicts' within \(\mathcal{K}\). A rather simple but still popular approach to measure inconsistency is thus taking the value \(\left| \mathcal{I}_{\min}(\mathcal{K}) \right|\). The notion of strong inconsistency facilitates the following generalization of this measure to arbitrary logics."

\(\mathcal{I}_{\text{MSI}}: 2^{\text{WF}} \rightarrow \mathbb{R}_{\geq 0}^{\infty} \text{ via } \mathcal{I}_{\text{MSI}}(\mathcal{K}) = \left| SI_{\min}(\mathcal{K}) \right|\)

\(\mathcal{I}_{\text{MSI}^\text{C}}: 2^{\text{WF}} \rightarrow \mathbb{R}_{\geq 0}^{\infty} \text{ via } \mathcal{I}_{\text{MSI}^\text{C}}(\mathcal{K}) = \sum_{\mathcal{H} \in SI_{\min}(\mathcal{K})} \frac{1}{|\mathcal{H}|}\)
\cite{hunter_measuring_2008}

\(\mathcal{I}_{\text{p}}: 2^{\text{WF}} \rightarrow \mathbb{R}_{\geq 0}^{\infty} \text{ via } \mathcal{I}_{\text{p}}(\mathcal{K}) = \left| \bigcup_{\mathcal{H} \in SI_{\min}(\mathcal{K})} \mathcal{H} \right|\)
\cite{liu_measuring_2011}

\section{Rationality Postulates}
"We develop rationality postulates based on previous ones from the literature [...] most of them require refinements" in Section 3 of \cite{ulbricht_measuring_2018} and Section 4 of \cite{ulbricht_handling_2020}
\\
"There is a growing number of rationality postulates for inconsistency measurement but not every postulate is generally accepted \cite{hameurlain_basic_2017} \cite{ferme_revisiting_2014}"
\\
"This becomes apparent when considering the monotony postulate which is usually satisfied by classical inconsistency measures and demands \(\mathcal{I}(\mathcal{K}) \leq \mathcal{I}(\mathcal{K}')\) whenever \(\mathcal{K} \subseteq \mathcal{K}'\) holds, i.e., the severity of inconsistency cannot be decreased by adding new information. However, in nonmonotonic frameworks, adding information may resolve conflicts. It is thus possible that \(\mathcal{K}\) is inconsistent, while \(\mathcal{K}'\) is not, so we would expect \(\mathcal{I}(\mathcal{K}') < \mathcal{I}(\mathcal{K})\) for any reasonable measure \(\mathcal{I}\)."

\subsection{Basic Postulates}

\subsection{Extended Postulates}

\section{Repair Knowledge Bases}
"We extend the hitting set duality from previous work \cite{brewka_strong_2019} to situations where knowledgebases can be repaired by adding information" in section 7 of \cite{ulbricht_handling_2020}
