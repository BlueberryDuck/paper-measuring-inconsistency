In knowledge representation and reasoning, dealing with inconsistent information is a challenge because classical logic considers any inconsistency as rendering a knowledge base useless. However, paraconsistent logic allows different inconsistent knowledge bases to convey varying amounts of information, leading to the need for inconsistency measures, which are quantitative metrics that assess the extent and impact of inconsistencies within a knowledge base. This paper explores possible inconsistency measures within non-monotonic logics, focusing on logic programming with Answer Set Programming (ASP) as an example. Applying the concept of strong inconsistency, which identifies subsets of knowledge bases that remain inconsistent even when adding more information, to measures based on minimal inconsistent subsets, these measures are adapted for non-monotonic logics. The paper further discusses revised rationality postulates to ensure theoretical soundness and practical applicability and assesses the adapted measures with these postulates.
