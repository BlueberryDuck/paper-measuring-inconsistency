\section{Appendix}

\begin{frame}{Semantics of Logic Programs}
    \begin{block}{Models and Answer Sets}
        \begin{itemize}
            \item A \textbf{model} \(M\) of a logic program \(P\) satisfies all rules in \(P\).
            \item \(M\) is a model if for every rule \(r\):
                  \begin{itemize}
                      \item If \(pos(r) \subseteq M\) and \(neg(r) \cap M = \emptyset\), then \(head(r) \cap M \neq \emptyset\).
                  \end{itemize}
            \item An \textbf{answer set} is a minimal model of \(P\).
        \end{itemize}
    \end{block}
    \begin{block}{Handling Negation}
        \begin{itemize}
            \item The presence of default negation (\texttt{not}) requires special treatment.
            \item Use the \textbf{Gelfond-Lifschitz reduct} to eliminate default negation when computing answer sets.
        \end{itemize}
    \end{block}
\end{frame}

\begin{frame}{Gelfond-Lifschitz Reduct}
    \begin{block}{Definition of Reduct}
        For a logic program \(P\) and a set of literals \(M\), the \darkhighlight{reduct} \(P^M\) is defined as:
        \[
            P^M = \{head(r) \leftarrow pos(r) \mid r \in P, \mathhighlight{neg(r) \cap M = \emptyset}\}
        \]
        \begin{itemize}
            \item All rules where the negative body is not contradicted by \(M\).
            \item Removes default negation to simplify the program.
        \end{itemize}
    \end{block}
    \begin{block}{Answer Sets via Reduct}
        A set \(M\) is an \darkhighlight{answer set} of \(P\) if \(M\) is a minimal model of the reduct \(P^M\).
    \end{block}
\end{frame}

\begin{frame}{Computing the Penguin Answer Set}
    \begin{block}{Candidate Answer Set \(M\)}
        \[
            M = \{\text{penguin(polly)},\ \text{bird(polly)},\ \text{ab(polly)}\}
        \]
    \end{block}
    \begin{block}{Constructing the Reduct \(P^M\)}
        \[
            P^M = \left\{
            \begin{array}{lcl}
                \text{penguin(polly)} &            &                       \\
                \text{bird(polly)}    & \leftarrow & \text{penguin(polly)} \\
                \text{ab(polly)}      & \leftarrow & \text{penguin(polly)}
            \end{array}
            \right\}
        \]
        \begin{itemize}
            \item The rule for \text{flies(polly)} is excluded since \(\text{not }\text{ab(polly)}\) is not satisfied.
        \end{itemize}
    \end{block}
\end{frame}

\begin{frame}{SI-Free Postulate Compliance}
    SI-free is not satisfied by any of the measures.
    \begin{exampleblock}{Example}
        Consider the logic program \(P = \{a \leftarrow \text{not } a, b., a \leftarrow \text{not }c., d \leftarrow \text{not }d., b., c., d.\}\), which has \(SI_{\min}(P) = \{\{a \leftarrow \text{not }a, b., b., c.\}\}\)
        and \(\alpha = d.\) and \(\beta = a \leftarrow \text{not }c.\) are in \(Free_{\text{SI}}(P)\).

        The measures evaluate to:
        \[
            \mathcal{I}_{\text{MSI}}(P) = 1,\quad \mathcal{I}_{\text{MSI}^\text{C}}(P) = \frac{1}{3},\quad \mathcal{I}_{\text{p}}(P) = 3.
        \]
        But if \(\alpha\) and \(\beta\) are removed, so \(SI_{\min}(P \backslash \{\alpha,\beta\}) = \{\{a \leftarrow \text{not }a, b., b.\}, \{d \leftarrow \text{not }d.\}\}\), they evaluate to:
        \[
            \mathcal{I}_{\text{MSI}}(P \backslash \{\alpha,\beta\}) = 2,\quad \mathcal{I}_{\text{MSI}^\text{C}}(P \backslash \{\alpha,\beta\}) = \frac{3}{2},\quad \mathcal{I}_{\text{p}}(P \backslash \{\alpha,\beta\}) = 3.
        \]
    \end{exampleblock}
\end{frame}
